\documentclass[10pt]{beamer}

\ifx\pdfoutput\undefined
% we are running LaTeX, not pdflatex
\usepackage{graphicx}
\else
% we are running pdflatex, so convert .eps files to .pdf
\usepackage{graphicx}
\usepackage{epstopdf}
\usepackage[level]{datetime}
\usepackage{enumerate}
\usepackage{amsmath}
\usepackage{amssymb}
\fi

\newdateformat{ukdate}{\ordinaldate{\THEDAY} \monthname[\THEMONTH] \THEYEAR}%������

\input{beihangbeamerstyle/beihangcolor}
\input{beihangbeamerstyle/beihangbeamerstyle}

\title{Spiking Neuron Model and Python Implement}
\author{Yan Zhou\\
Northeastern University (\texttt{zhouy@mail.neu.edu.cn})}
\date{Seminar, Shenyang, \ukdate\today}


\begin{document}
%----------------------------------------------------------------------
% Title frame
\begin{frame}[plain]
\maketitle
\end{frame}

%----------------------------------------------------------------------
% Outline frame
% PLEASE RUN pdflatex TWICE
\begin{frame}
\frametitle{Outline}
\tableofcontents
\end{frame}
%%=====================================================================
% Section I
\section{Neuronal Coding}
%----------------------------------------------------------------------
% Content frame
\begin{frame}\frametitle{Neuronal Coding}
%\framesubtitle{From the stimulus to Response}
\begin{block}{Introduction}
\begin{itemize}
\item Traditionally it has been thought that most, if not all, of the relevant information was contained in the mean firing rate of the neuron. Neuronal coding is a way of convert the pulses to analog values.
\item The relation between the measured firing frequency $v$ and the applied input current $I_0$ is sometimes called the {\color{red}{frequency-current curve}} of the neuron.
\item In models, we formalize the relation between firing frequency (rate) and input current and write $v = g(I_0)$. We refer to $g$ as the neuronal gain function or transfer function\cite{RN85}.
\item For instance, the quantity of electric charge $\varDelta tI_0=\int^{t+\Delta t}_t \sum_i e(\tau-t_i)\,d\tau$, and the firing rate satisfy $\varDelta tv=\int^{t+\Delta t}_t \sum_i \delta(\tau-t_i)\,d\tau$.
\end{itemize}
\end{block}
\end{frame}
%\pause
%----------------------------------------------------------------------
\begin{frame}
\begin{block}{Rate Codes}\frametitle{Neuronal Coding}
\begin{itemize} 
\item Average over time.
\item Average over several repetitions of the experiment.
\item Average over a population of neurons.
\end{itemize}
\end{block}
%\begin{block}{Another block}
%This block appears after a pause. Simply delete the \texttt{\textbackslash pause} command if this animation is not needed. Add the pause command whenever a pause is needed.
%\end{block}
\end{frame}
%----------------------------------------------------------------------
\begin{frame}\frametitle{Neuronal Coding}
\begin{block}{Rate as a Spike Count (Average over Time)}
\begin{equation}
\label{AOTe}
v=\frac{n_{sp}(T)}{T} = \frac {\int^T_0 \sum^{n_{sp}}_{i=1}\delta(\tau-t_i)d\tau}{T}
\end{equation}
\end{block}
\begin{figure}[htb]
\centering
\includegraphics[width=0.9\linewidth]{image/1.pdf}
\caption{A. Definition of the mean firing rate via a temporal average. B. Gain function, schematic. The output rate $v$ is given as a function of the total input $I_0$.}
\label{AOT}
\end{figure}
\end{frame}
%----------------------------------------------------------------------
\begin{frame}\frametitle{Neuronal Coding}
\begin{block}{Rate as a Spike Density (Average over Several Runs)}
\begin{equation}
\label{AOSe}
v=\frac {1}{\Delta t} \frac{n_{k}(t;t+\Delta t)}{K}
\end{equation}
\end{block}
\begin{figure}[htb]
\centering
\includegraphics[width=0.6\linewidth]{image/2.pdf}
\caption{Definition of the spike density in the Peri-Stimulus-Time Histogram(PSTH) as an average over several runs of the experiment.}
\label{AOS}
\end{figure}
\end{frame}

%----------------------------------------------------------------------
\begin{frame}\frametitle{Neuronal Coding}
\begin{block}{Rate as a Population Activity (Average over Several Neurons)}
\begin{equation}
\label{AOSNe}
v=\frac {1}{\Delta t} \frac{n_{act}(t;t+\Delta t)}{N} = \frac {1}{\Delta t}\frac{\int^{t+\Delta t}_t \sum_j\sum_f\delta(\tau-t_j^{(f)})d \tau}{N}
\end{equation}
\end{block}
\begin{figure}[htb]
\centering
\includegraphics[width=0.8\linewidth]{image/3.pdf}
\caption{A. A postsynaptic neuron receives spike input from the population $m$ with activity $A_m$. B. The population activity is defined as the fraction of neurons that are active in a short interval $[t; t + \Delta t]$ divided by $\Delta t$.}
\label{AOSN}
\end{figure}
\end{frame}
%----------------------------------------------------------------------
\begin{frame}
\frametitle{More Detailed Methods of Neuronal Coding}
\begin{columns}
\begin{column}{0.5\textwidth}
\begin{figure}[htb]
\centering
\includegraphics[width=1\linewidth]{image/4.pdf}
\caption{Firing rates approximated by different procedures\cite{RN113}.}
\label{A}
\end{figure}

\end{column}
\begin{column}{0.5\textwidth}
\begin{block}{Counting \& Convolution}
\alt<2>{
\begin{itemize}
\item $v_{approx}(t)=\sum^n_{i=1}w(t-t_i)=\int_{-\infty}^{+\infty} \sum^n_{i=1} d(\tau)w(\tau)\delta(t-\tau)$
\item $w(\tau) = \begin{cases} 1/\Delta t \quad if -\Delta t/2\geqslant t \ge \Delta t/2 \\0 \qquad otherwise.\end{cases}$
\item $w(\tau) = \frac {1}{\sqrt{2\pi}\sigma}exp(-\frac{\tau^2}{2\sigma^2})$.
\end{itemize}}
{\begin{enumerate}[(A)]
\item A spike train from a neuron.
\item Discrete time firing rate obtained by counting spikes 
\item Approximate firing rate determined by sliding a rectangular window function along the spike train. 
\item Approximate firing rate computed using a Gaussian window function. 
\item Approximate firing rate for an $\alpha$ function window.
\end{enumerate}}
%{\color{coreBlue}{This is the only ``Beihang colour'' I found. }} Other built-in colours can also be used, e.g. {\color{red}{red}}, {\color{orange}{orange}}, {\color{blue}{blue}}
%\vspace{9.5em}

\end{block}
\end{column}
\end{columns}
\end{frame}

%%=====================================================================
% Section II
\section{Formal Spiking Neuron Model}
%----------------------------------------------------------------------
\begin{frame}
\frametitle{\secname}
\begin{block}{Formal threshold models of neuronal firing}
Spikes are generated whenever the membrane potential $u$ crosses some threshold $\vartheta$ from below. The moment of threshold crossing defines the firing time $t(f)$.
\begin{equation}
\label{FSNM}
t^{(f)}: \quad u(t^{(f)})=\vartheta \quad and \quad \frac {du(t)}{dt}\big|_{t=t(f)}>0
\end{equation}
\end{block}
\end{frame}

%----------------------------------------------------------------------
\begin{frame}
\frametitle{\secname}
\begin{block}{Formal threshold models of neuronal firing}
\begin{itemize}
\item Integrate-and-fire model\cite{RN85}
    \begin{itemize}
    \item Leaky Integrate-and-Fire Model
    \item nonlinear Integrate-and-Fire Model
    \end{itemize}
\item Izhikevich Model\cite{RN96}
\end{itemize}
\end{block}
\end{frame}

%%=====================================================================
% Section III
\section{Spiking Neurons Design and Simulations}
%----------------------------------------------------------------------
\begin{frame}
\frametitle{Spiking Neurons Design and Simulations}
\begin{block}{Leaky Integrate-and-Fire Model}
\begin{figure}[htb]
\centering
\includegraphics[width=0.6\linewidth]{image/5.pdf}
\caption{}
\label{LIF}
\end{figure}


\end{block}
\end{frame}

%%=====================================================================
% Section IV
\section{Bibliography}
%----------------------------------------------------------------------
\begin{frame}
\frametitle{Bibliography}
\bibliographystyle{plain}
\bibliography{reference}
\end{frame}
\end{document}

